\documentclass[12pt]{article}
\usepackage[utf8]{inputenc}
\usepackage{mathtools}
\usepackage{amsfonts}
\begin{document}
\noindent
\section\\
Máme tedy klíče $k,l \in \mathbb{Z}_{27} \setminus \{0\}$ (těleso). Pokud by existovaly 2 zprávy a nějaká dvojice klíčů takové, že se zašifrují na stejný text, pak zobrazení $x \mapsto x \cdot k + l \mod 27$ není prosté pro libovolné $x$. Dokážeme, že je prosté:\\
$x,y \in \mathbb{Z}_{27}$\\
$x \mapsto x \cdot k + l \mod 27 $\\
$y \mapsto y \cdot k + l \mod 27 $\\
$x \cdot k + l \equiv y \cdot k + l \mod 27$\\ 
$x \cdot k \equiv y \cdot k  \mod 27$ a $NSD(k, 27) = 1$ tedy krátíme\\
$x \equiv y \mod 27$\\
$x,y<27 \rightarrow x=y$\\
Tedy zobrazení je prosté, tedy nemůže zašifrovat 2 různá písmena na to samé $\rightarrow$ ani text (posloupnost písmen)
\section\\
x - nezašifrované písmeno\\
první část\\
$x \mapsto x + a \mod 256 \ (\coloneqq y)$\\
druhá část \\
$y \mapsto y \cdot c + b \mod 256 \ (\coloneqq e)$\\
e - zašifrované písmeno\\
po dosazení za $y$ je to vlastně\\
$x \mapsto (x+a) \cdot c + b \mod 256 \ (\coloneqq e)$\\
Tedy funkci $Kas$ lze vyjádřit vzorcem\\
$Kas(x,a,b,c) = x \cdot c + (a \cdot c + b) \mod 256$\\
Tedy to je afinní šifra, kde jeden klíč je $c \in \mathbb{Z}^{\text{*}}_{256}$ a druhý je $(a \cdot c + b) \in \mathbb{Z}_{256}$\\
Takže to je obyčejná afinní šifra, akorát s trochu jiným klíčem. Bezpečnosti to teda určitě nepomůže.\\
\section\\
Mějme tedy množinu $\mathbb{A}$ velkých písmen anglické abecedy, tedy $\left\vert{\mathbb{A}}\right\vert = 26$.
Index koincidence textu $x$ délky $n$ se počítá jako $I_{c}(x)=\sum_{i \in \mathbb{A}} \frac{f_{i}\cdot(f_{i}-1)}{n\cdot(n-1)}$ z definice, kde $f_i$ značí počet výskytů písmene $i$ v textu $x$. Ale $\frac{f_i}{n}$ vlastně značí pravděpodobnost toho, že pokud vybereme náhodný znak z $x$, bude to $i$. Jelikož text je náhodný ze 26 znaků, potom ta pravděpodobnost je rovna $\frac{1}{26}$. Pokud uvažujeme délku text $n\rightarrow \infty$, tak je rozdíl $f_i$ a $f_i-1$ (i $n$ a $n-1$) zanedbatelný, tedy můžeme psát $I_{c}(x)=\sum_{i \in \mathbb{A}} {(\frac{1}{26})}^2$. Tato suma je konečná přes 26 prvků množiny $\mathbb{A}$ tedy $I_{c}(x)= 26\cdot{(\frac{1}{26})}^2 = \frac{1}{26}$. Tedy pro libovolný náhodný text $x$ platí $I_c(x) \approx \frac{1}{26}$
\end{document}
\documentclass[12pt, a4paper]{article}
\usepackage[utf8]{inputenc}
\usepackage{indentfirst} %indentace prvního odstavce
\usepackage{amsmath}
\usepackage{graphicx}
\usepackage{standalone}
\usepackage{preview}
\usepackage{xcolor}
\usepackage{mathtools}
\graphicspath{ {doc-img/} }

\begin{document}


\section{}
První náhodná veličina neexistuje, protože kód není prefix-free, takže nemůže být Huffmanův. 10 je prefixem 101.

Druhá náhodná veličina také neexistuje, protože kód také není Huffmanův, jelikož není "nejlepší možný". Nemá tedy nejmenší možnou entropii, protože kód 110 by šel zkrátit na 11 a vše by fungovalo.

Třetí kódování již Huffmanovo je. Nechť kódujeme množinu $A = \{1,2,3,4,5\}$  Náhodná veličina X: 
\begin{center}
\begin{tabular}{ |c|c|c|c|c|c| } 
\hline
$a \in A$ &1 & 2 & 3 & 4 & 5\\ 
\hline
Pr[X=a] & 0,78 & 0,05 & 0,05 & 0,06 & 0,06\\ 
\hline
Kód & 1 & 000 & 001 & 010 & 011\\ 
\hline
\end{tabular}
\end{center}

\section{}
Nechť platí $\exists x,y \in \mathcal{C}, z \in \{0,1\}^* \setminus \epsilon : x=y||z$. Z definice Huffmanova kódování je $y$ list ale $x$ je z definice také list. Ale při dekódování $x$, bychom se dostali nejdříve do $y$ (což je list) ale poté bychom nemohli již nikam pokračovat. Tedy by platilo, že $x=y$, což je ale spor, jelikož $z \neq \epsilon$.

\end{document}
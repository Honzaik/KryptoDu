\documentclass[12pt, a4paper]{article}
\usepackage[utf8]{inputenc}
\usepackage{indentfirst} %indentace prvního odstavce
\usepackage{mathtools}
\usepackage{amsfonts}
\usepackage{amsmath}
\usepackage{amssymb}
\begin{document}


\section{}
Rozebereme teda všechny různé funkce v lDES. Nechť $p_1,p_2,k,k_1,k_2 \in \mathbb{Z}_2^{64}$, operace $+ \iff \oplus$.

\textbf{IP}: je obyčejný výběr bitů (zbavení se parity bits) a jejich permutace. Vybeme-li tedy bity z $p_1, p_2$ a poté je sečteme je to samé jako jejich sečtení a poté výběr jejich součtů. Vždy se vyberou bity na stejných pozicích. Tedy $IP(p_1 + p_2)=IP(p_1)+IP(p_2)$. To samé pro její inverz.

Nyní budeme používat pro vstupy funkcí stejné značení jako u funkce IP. Samozřejmě do následujících funkcí není na vstupu přímo plaintext, ale jeho modifikovaná forma předcházejícími funkcemi. Stejně tak pro klíč.

\textbf{Split} (rozdělení na levou polovinu a pravou): Tato funkce je zřejmě také lineární, jelikož jednoduše rozděluje vstup na 2 poloviny. Je stejné jestli je nejdříve sečteme a rozdělíme nebo rozdělíme a poté půlky sečteme. Tedy $Split(p_1 + p_2) = Split(p_1)+Split(p_2)$.

\textbf{Expand} (expanze 32-bit vstupu na 48-bit): Zřejmě také lineární, jestliže bity sečteme a poté expandujeme bude mít stejný výsledek jako expanze $p_1, p_2$ a poté jejich následný součet (bity z $p_1, p_2$ jdou na stejnou pozici). $Expand(p_1+p_2)=Expand(p_1)+Expand(p_2)$.

\textbf{PC1}: Stejný případ jako \textbf{IP}, akorát je tato funkce aplikována na $k_1, k_2$. $PC(k_1+k_2)=PC(k_1)+PC(k_2)$.

\textbf{Shift}: Bitová rotace je také lineární. Jestliže posuneme bity $k_1$ a $k_2$ o $n$ bitů doleva/doprava a sečteme je, tak dostaneme stejný výsledek jako, když je sečteme a posuneme. $Shift(k_1+k_2,n) = Shift(k_1,n)+Shift(k_2,n)$.

\textbf{PC2}: Stejné jako \textbf{IP}, \textbf{PC1}.

\textbf{Add} (přičtení klíče): Tato funkce zřejmě není přímo z definice lineární (v plaintextu ani klíči), jelikož $Add(p_1+p_2,k)=(p_1+p_2)+k \neq (p_1+k)+(p_2+k) = Add(p_1,k)+Add(p_2,k)$ a $Add(p,k_1+k_2)=p+(k_1+k_2) \neq (p+k_1)+(p+k_2) = Add(p,k_1)+Add(p,k_2)$. Použijeme-li fakt, že funkce je lineární, pokud každý výstupní bit je lineární kombinací vstupních bitů, tak funkce \textbf{Add} je lineární. Výstupy $p_1+p_2+k $ a $p_1+p_2+2k$ (koeficienty jsou $(1,1,1)$ a $(1,1,2)$) jsou lineární kombinace vstupních bitů $p_1,p_2,k$. Výstupy $p+k_1+k_2$ a $2p+k_1+k_2$ (koeficienty jsou ($(1,1,1)$ a $(2,1,1)$) jsou lineární kombinace vstupních bitů $p,k_1,k_2$.

\textbf{Sbox}: Z předpokladů je lineární.

\textbf{P} (permutace výstupu z \textbf{Sbox}): Stejné jako \textbf{IP}, \textbf{PC1}, \textbf{PC2}

\textbf{RoundKey}: Odvození klíče je složení lineárních funkcí \textbf{PC1}, \textbf{Shift}, \textbf{PC2} a složení lineárních funkcí je lineární.

\textbf{F}: Feistelova funkce je složení lineárních funkcí \textbf{Expand}, \textbf{Add}, \textbf{Sbox} a \textbf{P}. Takže je také lineární.

\textbf{Round}: Runda je složena z lineárních funkcí \textbf{F}, \textbf{RoundKey} a přičtení výstupu z \textbf{F} k jedné polovině (v podstatě stejné jako v \textbf{Add}). Takže je také lineární.

Celkově je teda lDES lineární v klíči i plaintextu.



\section{}
Invert viz python kód.


Inverz \texttt{0xF3} v tělese $\mathbb{Z}_2[x]/(x^8+x^7+x^2+x+1)$ je \texttt{0x85}. Pokud tento prvek vynásobíme (jako vektor) maticí z AES dostaneme \texttt{0xEC} a po přičtení vektoru dostaneme výsledek \texttt{0x8F}.

\section{}
Ze schématu šifrování plyne $DES_{b}(c)=DES_{a}(p)$, kde $p$ je daný plaintext a $c$ daný ciphertext. Provedeme meet-in-the-middle útok. Počet různých klíču $a$ je díky jeho vlastnostem $(\frac{256}{2})^3=128^3\approx2\cdot 10^6$ (počet $k_i$ s lichou paritou je pouze polovina), což není tolik. Vygenerujeme všechny různé ciphertexty, které je možné získat šifrováním plaintextu $p$ klíčem $a$. Celkem nagenerujeme $\approx 2$ milionu ciphertextů, které si někam uložíme společně s příslušným klíčem $a$.

Poté provedeme druhou část útoku, která bude zase naopak šifrovat ciphertext $c$ různými klíči $b$, kterých je také zhruba 2 miliony. Ty si však nemusíme ukládat (ani pravděpodobně nevygenerujeme všechny 2 miliony). Pokaždé stačí zkontrolovat, jesliže příslušný zašifrovaný ciphertext již máme v tabulce. Pokud najdeme shodu, tak víme, jaké jsou oba klíče $a,b$. Tedy známe klíč $k$.\\
Výsledný klíč $a = \texttt{07:07:07:01:01:01:01:01}$\\
$b = \texttt{0B:0B:0B:01:01:01:01:01}$\\
Celkem tedy $k = \texttt{07:07:07:0B:0B:0B}$


Zbytek viz Java kód \texttt{main.java}

\section{}
Pro $k$ musí platit $k<=255 = \texttt{FF}_{16}$. To je maximální hodnota, která jde uložit do jednoho bajtu. Tedy pro šifry s blokem délky $>255$ bajtů tento padding nelze použít.

\section{}
Nechť $x,y$ jsou nějaké zprávy, nechť $|x|$ je délka zprávy $x$ (pro y stejně).

Buď $x\neq y$ a $|x|=|y|$, poté padding $p$ je stejný pro obě zprávy. Výsledné zprávy jsou tedy $x||p$ (zpráva $x$, ke které je přidán padding) a $y||p$. Ale $x \neq y$ z předpokladů, tedy nemůže platit, že se výsledné zprávy budou rovnat (tedy $x||p \neq y||p$).

Nebo $x\neq y$ a $|x| \neq |y|$. Poté každá zpráva má jiný padding $p_x \neq p_y$. Tedy zase nemůže platit, že $x||p_x \neq y||p_y$ (již poslední bajt zprávy je různý, protože padding má různou délku).

\section{}
Blok je tedy 8 bajtový, padding má délku tedy maximálně 8 bajtů. Zpráva má 16 bajtů. Tedy buď padding je délky 8 $\Rightarrow$ 8 z 16 bajtů zprávy je určeno. Celkový počet různých zpráv je $256^{16}$ a v tomto případě nám zbývá $256^8$ možností, jak vybrat prvních 8 bajtů zprávy. Pravděpodobnost toho, že náhodná zpráva bude mít správný padding délky 8 je $\frac{256^8}{256^{16}}=256^{-8}$.

Padding může mít délku $1\dots8$ bajtů. Výsledná pravděpodobnost je tedy:
$$\sum_{i=1}^8 256^{-i} \approx 0.4\; \%$$

Pokud náhodná zpráva má dobrý padding, tak nejpravdědpobněji bude mít padding délku 1 bajt, protože takových zpráv je nejvíce ($256^{15}$ z $256^{16}$). Pravděpodobnost takové zprávy je $\frac{1}{256}$.
\end{document}
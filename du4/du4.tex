\documentclass[12pt, a4paper]{article}
\usepackage[utf8]{inputenc}
\usepackage{indentfirst} %indentace prvního odstavce
\usepackage{mathtools}
\usepackage{amsfonts}
\usepackage{amsmath}
\usepackage{amssymb}
\begin{document}


\section{}
TODO

\section{}
Invert viz python kód.


Inverz \texttt{0xF3} v tělese $\mathbb{Z}_2[x]/(x^8+x^7+x^2+x+1)$ je \texttt{0x85}. Pokud tento prvek vynásobíme (jako vektor) maticí z AES dostaneme \texttt{0xEC} a po přičtení vektoru dostaneme výsledek \texttt{0x8F}.

\section{}
Ze schématu šifrování plyne $DES_{b}(c)=DES_{a}(p)$, kde $p$ je daný plaintext a $c$ daný ciphertext. Provedeme meet-in-the-middle útok. Počet různých klíču $a$ je díky jeho vlastnostem $(\frac{256}{2})^3=128^3\approx2\cdot 10^6$ (počet $k_i$ s lichou paritou je pouze polovina), což není tolik. Vygenerujeme všechny různé ciphertexty, které je možné získat šifrováním plaintextu $p$ klíčem $a$. Celkem nagenerujeme $\approx 2$ milionu ciphertextů, které si někam uložíme společně s příslušným klíčem $a$.

Poté provedeme druhou část útoku, která bude zase naopak šifrovat ciphertext $c$ různými klíči $b$, kterých je také zhruba 2 miliony. Ty si však nemusíme ukládat (ani pravděpodobně nevygenerujeme všechny 2 miliony). Pokaždé stačí zkontrolovat, jesliže příslušný zašifrovaný ciphertext již máme v tabulce. Pokud najdeme shodu, tak víme, jaké jsou oba klíče $a,b$. Tedy známe klíč $k$.\\
Výsledný klíč $a = \texttt{07:07:07:01:01:01:01:01}$\\
$b = \texttt{0B:0B:0B:01:01:01:01:01}$\\
Celkem tedy $k = \texttt{07:07:07:0B:0B:0B}$


Zbytek viz Java kód \texttt{main.java} + \texttt{DES.java}

\section{}
Pro $k$ musí platit $k<=255 = \texttt{FF}_{16}$. To je maximální hodnota, která jde uložit do jednoho bajtu. Tedy pro šifry s blokem délky $>255$ bajtů tento padding nelze použít.

\section{}
Nechť $x,y$ jsou nějaké zprávy, nechť $|x|$ je délka zprávy $x$ (pro y stejně).

Buď $x\neq y$ a $|x|=|y|$, poté padding $p$ je stejný pro obě zprávy. Výsledné zprávy jsou tedy $x||p$ (zpráva $x$, ke které je přidán padding) a $y||p$. Ale $x \neq y$ z předpokladů, tedy nemůže platit, že se výsledné zprávy budou rovnat (tedy $x||p \neq y||p$).

Nebo $x\neq y$ a $|x| \neq |y|$. Poté každá zpráva má jiný padding $p_x \neq p_y$. Tedy zase nemůže platit, že $x||p_x \neq y||p_y$ (již poslední bajt zprávy je různý, protože padding má různou délku).

\section{}
Blok je tedy 8 bajtový, padding má délku tedy maximálně 8 bajtů. Zpráva má 16 bajtů. Tedy buď padding je délky 8 $\Rightarrow$ 8 z 16 bajtů zprávy je určeno. Celkový počet různých zpráv je $256^{16}$ a v tomto případě nám zbývá $256^8$ možností, jak vybrat prvních 8 bajtů zprávy. Pravděpodobnost toho, že náhodná zpráva bude mít správný padding délky 8 je $\frac{256^8}{256^{16}}=256^{-8}$.

Padding může mít délku $1\dots8$ bajtů. Výsledná pravděpodobnost je tedy:
$$\sum_{i=1}^8 256^{-i} \approx 0.4\; \%$$

Pokud náhodná zpráva má dobrý padding, tak nejpravdědpobněji bude mít padding délku 1 bajt, protože takových zpráv je nejvíce ($256^{15}$ z $256^{16}$). Pravděpodobnost takové zprávy je $\frac{1}{256}$.
\end{document}
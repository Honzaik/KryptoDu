\documentclass[12pt, a4paper]{article}
\usepackage[utf8]{inputenc}
\usepackage{indentfirst} %indentace prvního odstavce
\usepackage{mathtools}
\usepackage{amsfonts}
\usepackage{amsmath}
\usepackage{amssymb}
\usepackage{graphicx}
\usepackage{color}

\graphicspath{{images/}}

\begin{document}

\section{}
Řetězce pro 56 bitů (14 znaků v hex) jsou:

\texttt{JanOupicky309152359}

\texttt{JanOupicky172522389}

 a příslušné SHA256 hashe:

\texttt{\textcolor{blue}{38852C61EE708A}5B14663408662333E92D28E56133F060A1E729EDE96CDE6544}

\texttt{\textcolor{blue}{38852C61EE708A}415601ED5E8B33081E2D0B7257B89ABDB8FAF5E16629C4DC78}

\section{}
Viz soubor \texttt{uloha2.py}. Klíč je \texttt{0x3412}.

\section{}
Nechť $j_k, i_k, S[j_k]$ značí hodnoty $i,j,S[j]$ na počátku smyčky v k. kroku. Počítáme v $\mathbb{Z}_{256}$. Platí tedy $j_k = i_k + 1$ a $S[j_k]=S[i_k + 1] = 1$. Potom:\\
\[i_{k+1} = i_k + 1\]
\[j_{k+1} = j_k + S[i_{k+1}] =  j_k + S[i_k + 1]\]
\[S[i_k + 1] = 1 \Rightarrow\ j_{k+1} = j_k + 1\]
\[j_k = i_k + 1= i_{k+1} \Rightarrow j_{k+1} = i_{k+1} + 1\]
\[S[j_{k+1}] \stackrel{\text{prohození}}{=} S[i_{k+1}] = S[i_k + 1] =1 \]

Tedy platí podmínky Finneyova stavu pro $k+1$. krok.

\section{}
Nechť Finneyův stav někdy nastane. Položme $k \geq 1$ (v 0. kroku je $i_0=0 \land j_0=0$, tedy nenastane Finneyův stav) první krok kdy Finneyův stav nastal (z předchozí úlohy víme, že takové k existuje). Dokážeme, že v tom případě musel nastat i v $k-1$. kroku $\Rightarrow$ spor s minimalitou $k$.

Zase počítáme v $\mathbb{Z}_{256}$.

Máme tedy $j_k=i_k+1 \land S[j_k]=S[i_k + 1]=1$. Víme, že v předchozím kroku nastalo: $j_k = j_{k-1}+S[i_k]$. Dále víme, že $S[i_k]$ se prohodilo s $S[j_k] = 1$, tedy $S[i_k] = 1$ taktéž.

Dále $i_k + 1 = j_k = j_{k-1} + S[i_k] = j_{k-1} + 1 \Rightarrow i_k = j_{k-1}$. Z definice $i_k = i_{k-1}+1 \Rightarrow j_{k-1}=i_{k-1}+1$. A také $S[j_{k-1}]=S[i_k]=1$. Takže i v $k-1$. kroku nastal Finneyův stav $\Rightarrow$ spor viz výše. Tedy Finneyův stav nikdy nemůže nastat.


\section{}
Ukážeme, že z kolize ($x$,$y$)  pro $h$ (tedy $x\neq y$) lze sestrojit kolizi pro $f$ (obměněná implikace).\\
Nechť $x = M_1||\dots||M_n$, $y = N_1||\dots||N_k$, $x' = M_1'||\dots||M_{n+1}'$ a $y' = N_1'||dots||N_{k+1}'$. Dále označme mezistavy pro $x$ jako $S_0, S_1, \dots, S_{n+1}$ a stavy pro $y$ jako $T_0, T_1, \dots, T_{k+1}$ kde $k,n \in \mathbb{N}$. Nakonec $q$ přísluší $x$ a $q'$ přísluší $y$. 

Předpokládáme, že: 
\[h(x)=h(y) \iff S_{n+1} = T_{k+1} \iff f(S_n, M_{n+1}') = f(T_k, N_{k+1}')\]
Takže buď $S_n \neq T_k \lor M_{n+1}' \neq N_{k+1}'$ a tím pádem máme kolizi pro $f$ (hotovo), nebo $S_n = T_k \land M_{n+1}' = N_{k+1}'$. Předpokládejme tedy, že platí $S_n = T_k \land M_{n+1}' = N_{k+1}'$, potom platí:
\[ [(M_{n+1}' = N_{k+1}') \land (M_{n+1}'=1||d-q) \land (N_{k+1}'=1||d-q')] \Rightarrow\ q=q'\]

Takže víme, že $q=q'$. Pokračujeme dále: 
\[S_n = T_k \iff f(S_{n-1}, M_n') = f(T_{k-1}, N_k')\]
Tedy zase buď máme kolizi pro $f$ (hotovo), nebo $S_{n-1} = T_{k-1} \land M_n' = N_k'$. Pokračujeme stejně:
\[[(M_n'=1||M_n||0^{d-q}) \land (N_k'=1||N_k||0^{d-q})] \Rightarrow\ M_n=N_k\]

Pokud $k=n$, tak jistě existuje $i \in \{1, \dots, n\}$ tž. $M_i' \neq N_i'$, protože $x \neq y$ $\Rightarrow$ kolize pro $f$.

Nebo BÚNO $k < n$. Pokud nenastane kolize pro $i \in \{2, \dots, k\}$, tak z toho plyne $N_1' = M_{n-k+1}'$, kde $N_1' = (0 || N_1) \land M_{n-k+1}'=(1||M_{n-k+1})$ což nemůže platit, protože se liší v prvním bitu. 

Tedy spor, takže musela nastat kolize pro $i\in\{2, \dots, k\}$, tedy kolize pro $f$. Takže $f$ bezkolizní $\Rightarrow h$ je bezkolizní.

\section{}
Víme $N = p \cdot q$ a $\varphi(N)=(p-1)\cdot(q-1)=p \cdot q - p - q + 1$.
\[N - \varphi(N) + 1 = p \cdot q - (p \cdot q - p - q + 1) + 1 = p + q \]
\[(p-q)^2 = p^2-2\cdot p \cdot q + q^2 = (p+q)^2 - 4\cdot p \cdot q \]
\[\Rightarrow p-q = \sqrt{(p+q)^2-4\cdot N} = \sqrt{(N - \varphi(N) + 1)^2 - 4\cdot N} \]
\[\Rightarrow p+q + (p-q) = 2\cdot p \Rightarrow p = \frac{p+q + (p-q)}{2}\]

A výsledek je:

\[p = 2611972000\dots 2808729943\]
\[q = 2633425658\dots 7502425297\]
\end{document}